% Fonts/languages
\documentclass[12pt,english]{exam}
\IfFileExists{lmodern.sty}{\usepackage{lmodern}}{}
\usepackage[T1]{fontenc}
\usepackage[latin9]{inputenc}
\usepackage{babel}
\usepackage{mathpazo}
%\usepackage{mathptmx}

% Colors: see  http://www.math.umbc.edu/~rouben/beamer/quickstart-Z-H-25.html
\usepackage{color}
\usepackage[dvipsnames]{xcolor}
\definecolor{byublue}     {RGB}{0.  ,30. ,76. }
\definecolor{deepred}     {RGB}{190.,0.  ,0.  }
\definecolor{deeperred}   {RGB}{160.,0.  ,0.  }
\newcommand{\textblue}[1]{\textcolor{byublue}{#1}}
\newcommand{\textred}[1]{\textcolor{deeperred}{#1}}

% Layout
\usepackage{setspace} %singlespacing; onehalfspacing; doublespacing; setstretch{1.1}
\setstretch{1.2}
\usepackage[verbose,nomarginpar,margin=1in]{geometry} % Margins
\setlength{\headheight}{15pt} % Sufficent room for headers
\usepackage[bottom]{footmisc} % Forces footnotes on bottom

% Headers/Footers
\setlength{\headheight}{15pt}	
%\usepackage{fancyhdr}
%\pagestyle{fancy}
%\lhead{For-Profit Notes} \chead{} \rhead{\thepage}
%\lfoot{} \cfoot{} \rfoot{}

% Useful Packages
%\usepackage{bookmark} % For speedier bookmarks
\usepackage{amsthm}   % For detailed theorems
\usepackage{amssymb}  % For fancy math symbols
\usepackage{amsmath}  % For awesome equations/equation arrays
\usepackage{array}    % For tubular tables
\usepackage{longtable}% For long tables
\usepackage[flushleft]{threeparttable} % For three-part tables
\usepackage{multicol} % For multi-column cells
\usepackage{graphicx} % For shiny pictures
\usepackage{subfig}   % For sub-shiny pictures
\usepackage{enumerate}% For cusomtizable lists
\usepackage{pstricks,pst-node,pst-tree,pst-plot} % For trees

% Bib
\usepackage[authoryear]{natbib} % Bibliography
\usepackage{url}                % Allows urls in bib

% TOC
\setcounter{tocdepth}{4}

% Links
\usepackage{hyperref}    % Always add hyperref (almost) last
\hypersetup{colorlinks,breaklinks,citecolor=black,filecolor=black,linkcolor=byublue,urlcolor=blue,pdfstartview={FitH}}
\usepackage[all]{hypcap} % Links point to top of image, builds on hyperref
\usepackage{breakurl}    % Allows urls to wrap, including hyperref

\pagestyle{head}
\firstpageheader{\textbf{\class\ - \term}}{\textbf{\examnum}}{\textbf{Due: Mar. 9\\ beginning of class}}
\runningheader{\textbf{\class\ - \term}}{\textbf{\examnum}}{\textbf{Due: Mar. 9\\ beginning of class}}
\runningheadrule

\newcommand{\class}{Econ 5253}
\newcommand{\term}{Spring 2021}
\newcommand{\examdate}{Due: March 9, 2021}
% \newcommand{\timelimit}{30 Minutes}

\noprintanswers                         % Uncomment for no solutions version
\newcommand{\examnum}{Problem Set 5}           % Uncomment for no solutions version
% \printanswers                           % Uncomment for solutions version
% \newcommand{\examnum}{Problem Set 5 - Solutions} % Uncomment for solutions version

\begin{document}
This problem set will give you practice with aggregating data from the web.

As with the previous problem sets, you will submit this problem set by pushing the document to \emph{your} (private) fork of the class repository. You will put this and all other problem sets in the path \texttt{/DScourseS21/ProblemSets/PS5/} and name the file \texttt{PS5\_LastName.*}. Your OSCER home directory and GitHub repository should be perfectly in sync, such that I should be able to find these materials by looking in either place. Your directory should contain four files:
\begin{itemize}
    \item \texttt{PS5\_LastName.R} (you can also do this in Python or Julia if you prefer)
    \item \texttt{PS5\_LastName.tex}
    \item \texttt{PS5\_LastName.pdf}
\end{itemize}
\begin{questions}
\question Type \texttt{git pull origin master} from your OSCER \texttt{DScourseS21} folder to make sure your OSCER folder is synchronized with your GitHub repository. 

\question Synchronize your fork with the class repository by doing a \texttt{git fetch upstream} and then merging the resulting branch. 

\question Find a webpage that has data that interests you, or that might be useful to you, but which doesn't have an API. Collect data from that webpage by following these steps:
\begin{itemize}
    \item Use SelectorGadget to select the CSS tags that correspond to the data you'd like to collect. This could be from anywhere---Wikipedia, a news website, a sports or pop culture website, etc.
    \item Use a package such as \texttt{rvest}, \texttt{BeautifulSoup}, or \texttt{Gumbo.jl} to parse the HTML and CSS code.
    \item Manipulate the parsed code so that you end up with a tabular data set (e.g. a data frame or a CSV file).
    \item In a new TeX file (\texttt{PS5\_LastName.tex}), tell me about the data. What about it interests you? Will it be useful to you at some point in the future? Did you use any online tutorials to help you parse the data? If so, which ones?
\end{itemize}

\question Find a website or other data source that hosts an API. For example: twitter, Quandl, Google Finance, or other sources. There exist many user-written packages in R, Python, and Julia that have set up the leg work for interfacing with more prominent APIs. 
\begin{itemize}
    \item Use the API to generate a table of data that is interesting/meaningful to you in some way.
    \item In another section of the same TeX file as referenced in the previous question, tell me about what you found in the data (if anything) that was interesting or useful to you. What packages did you use (if any)?
\end{itemize}

\question Compile your .tex file, download the PDF and .tex file, and transfer it to your cloned repository on OSCER using your SFTP client of choice (or via \texttt{scp} from your laptop terminal). You may also copy and paste your .tex file from your browser directly into your terminal via \texttt{nano} if you prefer, but you will need to use SFTP or \texttt{scp} to transer the PDF.\footnote{If you want to try out something different, you can compile your .tex file on OSCER by typing \texttt{pdflatex myfile.tex} at the command prompt of the appropriate directory. This will create the PDF directly on OSCER, removing the requirement to use SFTP or \texttt{scp} to move the file over.}

\question You should turn in the following files: .tex, .pdf,  and any additional scripts required to reproduce your work.  Make sure that these files each have the correct naming convention (see top of this problem set for directions) and are located in the correct directory (i.e. \texttt{\textasciitilde/DScourseS21/ProblemSets/PS5}).

\question Synchronize your local git repository (in your OSCER home directory) with your GitHub fork by using the commands in Problem Set 2 (i.e. \texttt{git add}, \texttt{git commit -m ''message''}, and \texttt{git push origin master}). Once you have done this, issue a \texttt{git pull} from the location of your other local git repository (e.g. on your personal computer). Verify that the PS5 files appear in the appropriate place in your other local repository.

\end{questions}
\end{document}
